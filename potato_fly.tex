\documentclass{ujarticle}
%\documentclass[twocolumn]{jarticle}
\usepackage[top=10truemm,bottom=15truemm,left=15truemm,right=15truemm]{geometry}
\usepackage[dvipdfmx]{graphicx}
\usepackage{float}
%\usepackage{caption}
%\captionsetup[table]{format=nonr, labelformat=simple, labelsep=period, font={sc, footnotesize}}
\renewcommand{\baselinestretch}{1.3}
\usepackage{amsmath}
\usepackage{cases}
\usepackage{ascmac}
\usepackage{comment}
\newcommand{\todaye}{令和元年\the\month 月\the\day 日}

\title{馬鈴薯の素揚げ}
\date{\todaye}

\begin{document}

\renewcommand\thefootnote{\arabic{footnote})}
\def\vector#1{\mbox{\boldmath $#1$}}

\maketitle

\begin{itembox}[l]{用意するもの}
    \begin{enumerate}
        \item 馬鈴薯
        \item 食用油
    \end{enumerate}
\end{itembox}

\begin{enumerate}
    \item 馬鈴薯を水で洗う。
    \item 馬鈴薯の皮を剥く。
    \item 馬鈴薯を4分の1にし、それを3分の1に切る。
    \item 鍋に水を入れ、切った馬鈴薯を入れ15分間煮る(弱火〜中火)。
    \item 茹で上がった馬鈴薯を取り出す。
    \item 鍋に油を入れ、油を加熱する。
    \item 油の温度が十分に高まったら、茹で上がった馬鈴薯を入れ素揚げする。
\end{enumerate}

\end{document}